\documentclass[a4paper,12pt]{article}

% Packages
\usepackage{amsmath, amssymb, amsthm}
\usepackage{graphicx}
\usepackage{hyperref}
\usepackage{enumitem}
\usepackage{xcolor}
\usepackage{tcolorbox}

% Custom Theorem Styles
\newtheorem{theorem}{Theorem}
\newtheorem{lemma}{Lemma}
\newtheorem{corollary}{Corollary}
\newtheorem{definition}{Definition}
\newtheorem{example}{Example}

% Custom Commands
\newcommand{\R}{\mathbb{R}}
\newcommand{\E}{\mathbb{E}}
\newcommand{\N}{\mathbb{N}}
\newcommand{\norm}[1]{\left\lVert #1 \right\rVert}

% Title Information
\title{Deep Learning Notes}
\author{Amir Nourinia}
\date{\today}

\begin{document}

\maketitle

\tableofcontents

\newpage

\section{Introduction}
This course is about introduction to Neural Networks and
it has a more hands on appraoch to the material. Each day two lectures are being held.
At the end of each day there is practical session to practice the material in the course.

\section{Neural Networks}

\subsection{Perceptron}
The perceptron is a simple linear classifier. It maps inputs to outputs. To increase its capabilities we
introduce a nonlinear function that acts upon the output of the normal perceptron. We call these nonlinear functions, Activation Functions.
There are many different activation functions that can we can use. Good activation functions should be fast to compute and easy to deriviate
and also it should be numerically stable.

\subsection{Activation Functions}
Some common activation functions are:
\begin{itemize}
  \item Sigmoid: $\sigma(x) = \frac{1}{1+e^{-x}}$
  \item ReLU: $ReLU(x) = \max(0, x)$
  \item Softmax: $\sigma(z_i) = \frac{e^{z_i}}{\sum_{j} e^{z_j}}$
\end{itemize}

\section{Training Neural Networks}

\subsection{Loss Functions}

\begin{tcolorbox}[colback=blue!5!white,colframe=blue!75!black,title=Loss Function]
  Loss Function is scalar method that assigns a loss value based on the current weights of the network.
\end{tcolorbox}

\subsection{Optimization Algorithms}

There are different algorithms for finding optimal weights for an assemble of neural networks. 
Note that our algorithm explores the space of losses generated by our training data.

\subsubsection{Gradient Descent}
\begin{tcolorbox}[colback=blue!5!white,colframe=blue!75!black,title=Gradient Descent]
  Gradient descent is an optimization algorithm used to minimize a function by iteratively moving 
  in the direction of the negative gradient.
\end{tcolorbox}

\subsubsection{Sarcastic Gradient Descent}
\begin{tcolorbox}[colback=blue!5!white,colframe=blue!75!black,title=Sarcastic Gradient Descent]
SGD 
\end{tcolorbox}


\subsection{Backpropagation}
The process of going back from the output towards inputs so that we can calculate the gradient of the 
loss function against a specific weight is called Backpropagation.

\subsection{Learning Rate}
The value of learning rate

\end{document}
